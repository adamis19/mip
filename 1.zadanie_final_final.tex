\documentclass{IEEEtran}     %dokument v strukture typu IEEEtran
\usepackage[IL2]{fontenc}      %zalamovanie
\usepackage[utf8]{inputenc}   %diakritika
\usepackage{graphicx}            %kniznica umoznujuca vkladat obrazky
\renewcommand\thesubsection{\Alph{subsection}}
\title{\large\textbf{Towards Collaborative Metadata Enrichment for Adaptive Web-Based Learning}}  %nadpis
\usepackage{dblfloatfix}

\author{                                                                                             %nadpis rozdeleny do autorov a ich adresy
\IEEEauthorblockN{Róbert Móro, Ivan Srba, Maroš Unčík, Mária Bieliková, Marián Šimko\\}
\IEEEauthorblockA{Institute of Informatics and Software Engineering, Faculty of Informatics and Information Technologies,
Slovak University of Technology, Ilkovičova 3, 842 47 Bratislava, Slovakia
{xmoror, xsrba, xuncik, maria.bielikova, marian.simko}@stuba.sk}}

\begin{document}           

\maketitle      %vypise titulok

\begin{abstract}
In recent years we have witnessed expansion of Web
2.0. Its main feature is allowing users’ collaboration in content
creation using various means, e.g. annotations, discussions,
wikis, blogs or tags. This approach has influenced also webbased
learning, for which the term “Learning 2.0”  has been
introduced. In this paper we explore using tags in such
systems. Tags can be used for improving of searching,
categorization of web-documents, creating folksonomies and
ontologies or enhancing the user-model. Another aspect of tags
is that they act as a bridge between resources and users to
create a social network. We integrated tags in a learning
framework ALEF and experimentally evaluated their usage in
education process.

%\keywords{Web collaboration; folksonomy; tag; learning framework}  keywords mam problem skompilovat
\end{abstract}

\section{introduction and related works}  %sekcia 1.urovne
An unprecedented expansion of web applications can be
observed since September 2005 when official definition of
the term Web 2.0 was published by O'Reilly \cite{o2007web}. During
this time, the main principles of Web 2.0 proved to be very
successful. Similar changes occurred also in the field of elearning.
In the same year Downes specified the term Elearning
2.0 \cite{downes2005learning}  but progress of technologies in e-learning
systems is not as fast as in the Web 2.0.

We research how to extend such systems effectively with
Users Add Value rule, which is one of the key principles of
Web 2.0. Users can be involved into the content creation in
many ways \cite{rollett2007web} and participate directly in the content
creation, i.e., in wiki systems, or can be involved in adding
potentially valuable metadata such as annotations or tags.

In this paper we focus on collaborative tagging in
learning systems and the characteristics of the social network
created by tags. We provide several recommendations how
to effectively integrate a social network acquired from tags
into learning systems. We also present an evaluation of our
experiment in real educational settings.

\subsection{Collaborative Tagging}   %sekcia druhej urovne
Although the idea of tagging is not new, it has increased
in popularity with the arrival of social bookmarking service
Delicious and photo sharing service Flickr. These services
popularized using of tags by adding the social aspect to the
process of tagging. Much research has been done in this
domain in recent years. Golder and Huberman \cite{golder2006usage} studied the
social bookmarking service Delicious, identified several
motives why users use tags and discovered stable patterns in
tag usage. They also discussed linguistic problems of tags
such as polysemy, synonymy and basic level variations.
The term folksonomy is strongly related to collaborative
tagging. It has been coined by Thomas Vander Wal in a
discussion on an information architecture mailing list, in
order to name the system of organization that emerges when
collaborative tagging is applied. This term originated as a
combination of folk and taxonomy and reflects the main
principle of collaborative tagging – people (users) use their
own vocabulary to categorize web documents. Mathes \cite{mathes2004folksonomies}
argues that despite folksonomy's problems, namely
ambiguity and synonymy, it works, mainly due to the ability
to directly reflect the vocabulary of users and it is also much
cheaper – both in terms of time and effort – than building the
complex classification hierarchy.

Collaborative tagging is one of the characteristics of the
Web 2.0 and also e-learning 2.0, which tries to take
advantage of the principles of the social web. This feature
has been recently integrated to learning frameworks such as
Moodle, Elgg or Blackboard (in its variation of social
bookmarking, Bd Scholar).

Torniai et al. designed a method for leveraging
folksonomies for learning \cite{torniai2008leveraging}. Their LOCO-Analyst tool
provides teachers with means for visualization of a domain
model associated with a selected learning object as well as a
tag cloud. Using Context-Based Relatedness Measure it
suggests user-provided tags related to a selected concept to
teachers, which can be used to update an existing ontology.
Through tags, teachers can also find out that students'
perception of the course content differs from the course
conceptualization encoded in domain ontology.

Tags can be categorized as a special type of annotations.
In this paper, by annotations we mean various remarks that
are related to a resource. Many web-based systems support
various types of annotations. Much work has been devoted to
educational domain. Tags and other types of annotations,
such as comments, bug reports or external sources, are often
used to enrich educational texts. Current educational systems
use automatic annotation frequently and at the same time
they allow users to add content by themselves. The common
way of automatic annotation is to use techniques for
information extraction and semantics discovery \cite{brut2008ensuring}. We are,
however, interested in the enrichment of content by users,
which is a different approach. New content is added by the
users and relies only on the users. This feature is employed
in several existing web-based educational systems, such as
AnnotatEd \cite{farzan2008annotated}.


\begin{figure*}[b]
\centering
\includegraphics[width=0.8\linewidth]{images/screen.png}
\caption{Example screenshot of ALEF (in Slovak). Recommender with recommendations is situated over the content navigation (1). Collaborative tools are visualized on the left – motivation tool (3), external resources inserter (4) and tagger (5) allow students to collaborate. Content can be enriched by adding different types of annotations and is accessed by pop-up menu over the highlighted sections of text (2).}
\label{screen}
\end{figure*}

\subsection{Social Networks}
Online social networks are already a very important part
of many popular web sites. Social networking sites often
offer a possibility of tagging, which allows users to
interactively annotate resources using descriptive tags. Thus,
the tag as a kind of social annotation can be used to create
social network. Although there are some research works on
this topic, a little attention is paid to investigate this area in
the domain of learning.

Wu and Zhou consider tags in social bookmarking sites
as a bridge in three dimensions: tag connects users and
resources, tag connects resources and tag connects users
\cite{wu2009analysis}. In this tag-centric approach the social network is
formed by users linked together by collaborative tagging
through sharing of resources. This way it is relatively
straightforward to find users' interests via shared tags. Such
a social network of tags has the feature of small world and a
scale-free network. Their results also showed that connected
tags had relatively strong semantic relatedness.

Mika formulated a model of semantic-social networks in
the form of tripartite graph of persons, concepts and instance
associations, extending the traditional concept of ontologies
with social dimension \cite{mika2005ontologies}. This representation allowed him
to derive a social network of persons based on overlapping
sets of tagged objects and a semantic network based on
community relationships.

In this paper, we focus on collaborative tagging in
learning, because though it is an active area of research and
tagging proved to be very successful in other domains, their
full potential in the domain of learning is yet to be explored.
An important difference with a general purpose tagging
system is that tags in these systems reflect not only user
interest in a subject, but also his or her knowledge. Thus, we
believe that tags can be used to derive domain model, or to
find users (students) with similar knowledge (and interests),
which is important for creating virtual study groups.

\section{tags in adaptive learning framework alef}
In order to provide students with a truly interactive
environment where they could participate in content creation
and collaborate, an adaptive learning framework ALEF has
been designed and implemented \cite{vsimko2010alef}. ALEF adapts main
principles of adaptive web-based learning. It is designed with
flexibility to develop loosely coupled components as one of
primary goals. ALEF represents a full-featured e-learning
portal that supports learning and collaboration/creation flow
\cite{bielikova2010personalized} currently used for learning programming (C, lisp and
prolog languages) and principles of software engineering.

The collaboration/creation flow allows adding various
types of content by using collaborative content creator tools
family. Currently, three collaborative content creators are
implemented – commentator, external resources inserter and
questions creator (see Fig.\ref{screen}, which shows two of them). We
have extended this family of tools by implementing the
tagger as is also shown in Fig. \ref{screen}.

Domain model of ALEF is divided into two layers:
content and metadata. The content consists of documents,
which can be accessed via an URI and can be presented to
the users. In order to describe the characteristics of learning
objects, the metadata layer is used. Metadata includes
annotations and concepts – knowledge domain elements.
ALEF's data model is sufficiently general to ease the adding
of new components such as collaborative content creators,
which was with advantage used in previous studies \cite{unvcik2010annotating}.

We have taken this advantage also for implementing the
tagger as a component in ALEF. From data model design
view we consider the tags as a special kind of annotations.
ALEF data model had already contained data entity for
annotations, as well as for special kinds of annotations (i.e.
comments, bug reports); we have easily extended the data
model using inheritance by creating a new special data entity
for tags (see Fig. \ref{schema}).

\begin{figure}[h]    %vycentrovany obrazok s popisom a sirkou polovice sirky textu
\centering
\includegraphics[width=0.5\textwidth]{images/schema.png}
\centering
\caption{ Fragment of ALEF data model relevant to tags (our extension of model is at the bottom).}
\label{schema}
\end{figure}

Using the tagger users can add tags to learning objects,
such as explanatory texts, questions or exercises. Tag can be
a keyword or a multiword phrase, which is limited only by
maximal character length. A user chooses, whether he wants
a tag to be \textit{private} (other students will not see the tag) or
\textit{anonymous} (tag will be public, but others will not see by
whom it was added). Users can view either tags they added
themselves (in \textit{my tags view}) or \textit{popular} tags (in \textit{popular
tags view}). A tag is considered popular in a learning object,
if it was added by at least three users as an anonymous tag to
the learning object. Users can also delete tags, but only those
they added themselves.

Finally, tags serve as a means of navigation in learning
objects space – when a user selects a tag, a list of all learning
objects tagged with the tag selected is shown. List of
learning objects shown depends on the tag view too. In \textit{my
tags view}, only learning objects tagged by the user are
shown, while in popular tags view, all learning objects
tagged with popular tags are shown.

\section{evaluation}
We have evaluated tagging behavior during an
uncontrolled experiment. We have assessed how students use
tags in ALEF and the capability of these tags to cover
important concepts of learning objects. Students were asked
and motivated to use ALEF and to tag learning objects to
prepare themselves for final exam in programming course.
We consider a social network of users created implicitly by
their collaboration and explicitly by their grouping into
courses, setups or virtual study groups.

The experiment lasted for 15 days. During this period we
gathered 2 272 tags, out of which 947 were unique. In order
to be able to evaluate these tags, we further normalized them
as follows: we removed diacritic marks and some other
characters, e.g. brackets that occur together with function
names (for example \textit{malloc()} to \textit{malloc}), converted all tags to
lowercase, lemmatized them and finally we translated some
frequently occurring tags (such as string or structure) from
English to their Slovak equivalents (\textit{retazec, struktura} resp.).
After normalization, 755 unique tags were left.

From all the tags gathered, only 2\% were deleted by
students and only 15\% were private. It shows, that students
used tags to describe the content of the learning objects and
not as a means for personal notes (for example, only two
ToDo tags were added). Each unique tag has 2.4 occurrences
on average; the overall distribution of tag occurrences
follows the power law distribution as expected (see Fig. 3),
when only one tag occurred 96 times, but 461 tags occurred
one time. Tab. 1 shows the number of top 5 tags'
occurrences (after normalization). We can see that these tags
well reflect important concepts in the domain of C
programming language, which was taught in the course.

\begin{figure}[h!!]
\centering
\includegraphics[width=0.5\textwidth]{images/tag.png}
\caption{ Distribution of tag occurrences. }
\label{tag}
\end{figure}

\begin{table}[h]           %vycentrovana tabulka
\centering
\caption{top 5 tags occurrences}  %popis
\begin{tabular}{|c|l|l|l|}                      %4 stlpce, jeden vycentrovany, tri zarovnane left
\hline                                                     %horizontalna ciara
\textbf{} & \multicolumn{1}{p{2.3cm}|}{\textbf{Tag (in Slovak, normalized)}} & \multicolumn{1}{p{2.3cm}|}{\textbf{Tag (translated to English)}} & \multicolumn{1}{p{2.3cm}|}{\textbf{Occurrence count}} \\ \hline
1         & pointer                                                   & pointer                                                   & 96                                             \\ \hline
2         & makro                                                     & macro                                                     & 57                                             \\ \hline
3         & retazec                                                   & string                                                    & 52                                             \\ \hline
4         & struktura                                                 & structure                                                 & 52                                             \\ \hline
5         & operator                                                  & operator                                                  & 42                                             \\ \hline
\end{tabular}
\end{table}

Next, we have explored the distribution of tags regarding
the learning objects. Students managed to tag 613 learning
objects (more than 77\%) with 3.63 tags in average
(considering only learning objects tagged by at least one tag).
The overall distribution again follows the power law (see
Fig. \ref{lo}), when one learning object was tagged with 28 tags,
while 131 of them were tagged only by one tag. 182 learning
objects (less than 23\%) were not tagged; these are not
considered in our evaluation.

\begin{figure}[h]
\centering
\includegraphics[width=0.5\textwidth]{images/lo.png}
\caption{Distribution of tags in regard to learning objects (LO).}
\label{lo}
\end{figure}

We have also evaluated students' behavior when adding
the tags. However, this particular evaluation has been
influenced by the fact that only 35 out of 82 students using
the system at the time of our experiment added tags.
Therefore, the average number of tags per student is 63.6.
Number of tags varies significantly from student to student
and the distribution does not follow the power law. Users
were allowed to add phrases instead of a keyword as a tag 
(up to 30 characters). However, almost 78\% of tags consisted
of only one word and only 16\% of two words. Average
number of words per tag was 1.28 and no tag consisted of
more than five words.

Finally, we were interested in the capability of tags to
cover important concepts defined by experts in the domain
model of the course. To calculate the overlay ratio  $\varphi$, we
compared normalized tags to normalized concepts associated
with the same learning object using (\ref{vzorec}), where |LO| is a
number of tagged learning objects, Tags(lo) and
Concepts(lo) are sets of tags and concepts associated with
the learning object lo respectively.

\begin{equation}    %vzorec 
\label{eq:1}
\varphi = \frac{\sum_{lo=1}^{|LO|} |Tags(lo)\cap Concepts(lo)|}{\sum_{lo=1}^{|LO|} |Concepts(lo)|}
\label{vzorec}
\end{equation}

We have found out that the overlay ratio $\varphi$  is 27.76\%. In  %vlozeny matematicky symbol, vzorec
the explored domain model, 263 unique concepts were
identified by the experts. From these, 49.8\% were covered
by the folksonomy, i.e. about half of the domain concepts
have an exact match (after normalization) with one of the
tags. This measure does not reflect relations between
concepts and learning objects, but rather indicates the
capability of folksonomy to identify important concepts in
the domain. Interestingly, only 17\% of unique tags added by
users were covered by the concepts. This can be partially
explained by the imperfection of our comparison measure or
by the use of semantically similar words, which were not
fully covered by our normalization method. However, many
of these tags were actually concepts which were not
originally contained in the domain model.

\section{conclusion}
In this paper, we have explored the characteristics of
personalized web-based learning 2.0 systems and focused on
the role of collaborative tagging in these systems. Our main
contribution lies in demonstrating how the tagging can easily
be introduced to an e-learning framework as an extension to
existing domain model. We have designed and implemented
tagger component in ALEF and experimentally evaluated
tagging during programming course in a real world setting.

Findings of our experiment suggest that it is possible and
beneficial to use folksonomies for domain modeling, either
to enrich an existing model or even to build it from the
bottom up. This approach is collaborative and social, because
it relies on users of the system to find consensus on the
concepts and their relationships are derived by co-occurrence
of tags. Thus, semantic network of the domain is derived by
social network of users. However, this process is
bidirectional and semantic network emerging from usage of
tags can enrich the social network of users (students).

As future work we intend to utilize folksonomy for this
purpose and explore other possible usages of collaborative
tagging, i.e. enriching the user model following term-based
user modeling approach \cite{barla2011towards} or providing personalized search
or automatic text summaries in the learning environment.

\section*{acknowledgement}

This work was partially supported by the grant KEGA
028-025STU-4/2010 and it is the partial result of the
Research \& Development Operational Programme for the
project Research of methods for acquisition, analysis and
personalized conveying of information and knowledge,
ITMS 26240220039, co-funded by the ERDF.

The authors wish to thank all members of the PeWe
group (pewe.fiit.stuba.sk) for their invaluable contribution to
discussions, development of the ALEF framework and
support in experiments. The most current state of ongoing
projects within the group is reported in \cite{Moro:2011:TCM:2052137.2052242}.

\bibliographystyle{abbrv}  %styl bibliografie
\bibliography{zdroje}  %sem vlozi bibliografiu zo suboru zdroje


\end{document}